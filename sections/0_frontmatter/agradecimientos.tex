\chapter{Agradecimientos}

\chapterQuote{\textit{``No hay nada que pueda tener en más alta estima que ser y mostrarse agradecido. Porque esta virtud no sólo es la más grande, sino que también es la madre de todas las demás virtudes.''}}{--- Marco Tulio Cicerón}

\chapterAbstract{A}{ pesar de que ésta es una de las secciones que encabeza el presente trabajo, fue la última en ser escrita. Tal vez porque, en cierto modo, es la más importante de todas. Me gustaría comenzar estas líneas dando las gracias a mis directores de tesis, David e Inma, por haberme brindado la oportunidad de unirme a su grupo de investigación hace ya algunos años, y por su guía y apoyo constante durante todo este proceso. Sobre todo, me gustaría reconocer públicamente su ejemplo de integridad académica y personal, y el cariño sincero con el que tratan a cualquiera bajo su supervisión. Particularmente, quiero agradecer a David la confianza que ha puesto en mí y en mis capacidades en una gran multitud de ocasiones, que espero haber correspondido; y a Inma, su apoyo y disponibilidad constantes, incluso en momentos en los que a duras penas tenía el tiempo para ello (y por sus muchas y excelentes recomendaciones de restaurantes para nuestras comidas de grupo). Ha sido un auténtico placer y un privilegio trabajar con ellos.\\

Me gustaría también agradecer a mis colegas ---en todas sus acepciones--- de Geozoco, Miguel y Fernando (y también a Paula y Pepe, que se han unido más recientemente) por todos los momentos y risas que hemos compartido juntos. Quizá su compañía ha hecho que trabaje algo menos de lo que debería, pero es el precio a pagar por tener tan buenos compañeros. Aunque tomemos caminos separados, estoy convencido de que encontrarán el éxito (y, más importante, la felicidad) en cualquier rumbo que decidan tomar. Tengo también la tranquilidad de saber que los que decidan iniciar su doctorado, y los que ya lo han hecho, no podrían estar en mejores manos.\\

Quiero también agradecer al Dr. Francesco Osborne por gestionar y facilitar las estancias de investigación que he realizado a lo largo de mis estudios de doctorado, y al resto de sus compañeros: Dr. Danilo Dess{\`i}, Dr. Diego Reforgiato Recupero, y Dr. Davide Buscaldi, por su colaboración continuada que aún se extiende al día de hoy, por compartir su gran dominio de los grafos de conocimiento científicos y muchos otros temas con nosotros, y por todo el gran trabajo que hemos hecho juntos.\\

Debo también agradecer al sistema público de educación y becas que, con sus muchos fallos e imperfecciones, me ha permitido tanto a mí como a muchos de mis compañeros acceder a una educación de alta calidad y alcanzar nuestro máximo potencial sin prácticamente coste, con independencia de nuestras posibilidades económicas. Me gustaría también agradecer a los muchos profesores que he tenido, tanto en la universidad como fuera de ella, ya que sin ellos nunca habría podido llegar hasta aquí. En el transcurso de mi beca de doctorado, he tenido el privilegio de enseñar en las mismas aulas donde yo mismo fui un estudiante no muchos años atrás. Creo firmemente que los mejores profesores son los que saben que siempre tienen algo nuevo que aprender de todo el mundo, por lo que quiero también agradecer a los estudiantes que han pasado por mis clases en estos años, de los que he aprendido mucho. Espero haber sido el profesor que se merecen.\\

Estas últimas líneas que escribo están reservadas, como no podría ser de otro modo, para las personas más importantes en mi vida: las que me hacen sentir en casa independientemente de dónde esté. A mis abuelos Manolo y Loli, que siempre hicieron lo imposible y más para que tuviéramos todo lo que necesitábamos. A mis padres Javier y Matilde, y mi hermano Caleb, por su amor y apoyo constante, inacabable e incondicional. Y a Ren, por una infinidad de motivos que no podrían, y no necesitan, ser enumerados aquí.
}
