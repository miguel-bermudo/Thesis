\chapter{Resumen}\label{chap:resumen}

\chapterQuote{\hfill\textit{``Una síntesis vale por diez análisis.''}}{--- Eugeni d'Ors}

\chapterAbstract{L}{os grafos de conocimiento han estado a la vanguardia del almacenamiento de información de dominio desde su creación. Estos grafos pueden servir de base para una serie de aplicaciones inteligentes, como la respuesta a preguntas o las recomendaciones de productos. Sin embargo, por lo general se construyen de forma automatizada y no supervisada, lo que a menudo da lugar a que falte información, normalmente en forma de enlaces que faltan entre entidades relacionadas en la fuente de datos original, y que tienen que añadirse a posteriori mediante técnicas de compleción.

La compleción de grafos de conocimiento trata de encontrar los elementos que faltan en un grafo de conocimiento, normalmente aristas que representan alguna relación entre dos conceptos. Una posible forma de hacerlo es encontrar caminos entre dos nodos que indiquen la presencia de una arista que falta. Esto puede lograrse mediante el Aprendizaje por Refuerzo, entrenando a un agente que aprenda a navegar por el grafo, comenzando en un nodo con una arista ausente e identificando qué arista de entre las disponibles en cada paso es más prometedora para alcanzar el objetivo de la arista ausente.

Aunque se han propuesto algunos enfoques en este sentido, sus funciones de recompensa sólo tienen en cuenta si se ha alcanzado o no el nodo objetivo, y sólo aplican un único algoritmo de Aprendizaje por Refuerzo. En este sentido, presentamos una nueva familia de funciones de recompensa basadas en la incrustación de nodos y la distancia estructural, aprovechando información adicional relacionada con la similitud semántica y eliminando la necesidad de alcanzar el nodo objetivo para obtener una medida de los beneficios de una acción

SpaceRL es un marco integral en Python diseñado para la generación de agentes de aprendizaje por refuerzo (RL), que pueden utilizarse para completar grafos de conocimiento y descubrir enlaces. El objetivo de los agentes generados es ayudar a identificar los enlaces que faltan en un grafo de conocimiento encontrando caminos que conectan implícitamente dos nodos, proporcionando de paso una explicación razonada del nuevo enlace inferido. La generación de tales agentes es una tarea compleja, más aún para un usuario no experto, y hasta donde sabemos no existen herramientas que proporcionen ese tipo de ayuda.

SpaceRL pretende superar estas limitaciones proporcionando un conjunto flexible de herramientas diseñadas con una amplia variedad de opciones de personalización, con el fin de ser lo suficientemente flexible como para adaptarse a las diferentes necesidades de los usuarios. También incluye una variedad de algoritmos RL de última generación y varios modelos de incrustación que pueden combinarse para optimizar el rendimiento del agente. Además, SpaceRL ofrece diferentes interfaces para que esté disponible de forma local (mediante programación o a través de una GUI), o a través de una API REST compatible con OpenAPI.}