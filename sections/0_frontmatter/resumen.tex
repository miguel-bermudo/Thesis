\chapter{Resumen}\label{chap:resumen}

\chapterQuote{\hfill\textit{``Lo bueno, si breve, dos veces bueno.''}}{--- Baltasar Gracián}

\chapterAbstract{H}{oy en día, los grafos de conocimiento son una herramienta ampliamente usada para almacenar y representar información estructurada para una gran variedad de dominios y aplicaciones prácticas. Sin embargo, debido a que generalmente son construidos usando técnicas de extracción automática de información, éstos suelen estar incompletos. Esto se debe a que las citadas técnicas pueden no extraer satisfactoriamente la información deseada, o a que la fuente original no contenía suficiente información.

El problema tratado en esta tesis doctoral es cómo encontrar este conocimiento que falta y completar un grafo de conocimiento de manera automática. En la bibliografía existen numerosas propuestas para lograr este objetivo, pero tienen importantes inconvenientes, concretamente: necesitan utilizar \textit{embeddings}, que son computacionalmente costosos de obtener y requieren ser regenerados frecuentemente, necesitan intervención humana o datos generados manualmente, tienen una dependencia fuerte con fuentes externas de información, no tienen ningún modo para generar nuevo conocimiento por ellas mismas, o no son aplicables a grafos de conocimiento muy grandes. 

En esta tesis presentamos una nueva propuesta automatizada para completar grafos de conocimiento que no sufre de los problemas anteriores. Nuestra contribución tiene tres elementos principales: CHAI, una técnica para generar automáticamente conjuntos manejables de tripletas candidatas; CAFE, una propuesta de clasificación de tripletas de alta precisión; y SciCheck, una técnica especialmente diseñada para completar grafos de conocimiento científicos. Nuestra validación, tanto teórica como basada en una aplicación práctica, sugiere que nuestra propuesta es muy eficiente y efectiva en casos de uso reales, y que es capaz de completar satisfactoriamente grafos de conocimiento de todo tipo.}