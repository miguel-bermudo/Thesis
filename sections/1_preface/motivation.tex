\chapter{Motivation}\label{chap:motivation}

\chapterQuote{\textit{``The worthwhile problems are the ones you can really solve or help solve, the ones you can really contribute something to. No problem is too small or too trivial if we can really do something about it.''}}{--- Richard Feynman}

\chapterAbstract{D}{}

% espite Knowledge Graph completion being a very active research topic, the current proposals for carrying it out still have some drawbacks that hinder their applicability in practice, and which should be solved. Our goal in this chapter is to present the problems that arise in practice when completing Knowledge Graphs, and to motivate the need for a new proposal. This chapter is organized as follows: Section~\ref{sec:moti-intro} introduces it and provides the necessary background knowledge, Section~\ref{sec:moti-problems} presents the problems of KG completion in detail, Section~\ref{sec:moti-analysis} analyzes the current approaches and their main problems, Section~\ref{sec:moti-discussion} explains how none of the existing proposals solves all practical problems at a time, Section~\ref{sec:moti-proposal} introduces our contributions and compares them with the existing proposals in the literature; finally, Section~\ref{sec:moti-summary} summarizes the chapter.



\section{Introduction}\label{sec:moti-intro}
\begin{itemize}
    \item Increase in interest by companies like google amazon etc\dots
    \item they are good for organization of information
    \item they need to be as complete as possible
    \item they need to be organized in such a way that the answers they provide have an explanation (explainability)
\end{itemize}
% Nowadays, there is an increasing interest of individuals, organizations and
% companies in Knowledge Graphs, in order to organize, store and publish their data. This, in turn, can support many practical applications in a variety of domains, such as commerce \cite{zhang2021, palumbo2020}, education \cite{chen2018, aliyu2020}, research \cite{wang2018, dessi2020aikg, dessi2022cskg}, or healthcare \cite{zhang2020b, gong2021, wise2020}, to cite a few.

% Ideally, these Knowledge Graphs should be as complete as possible, to make sure that they include all pieces of knowledge that may be relevant to the organization that manages it or the users they support. However, due to the way in which they are constructed, it is well-known that this is not the case \cite{paulheim2017,shen2022overview,hogan2020,peng2023}. Those that are automatically built from external knowledge sources rely on the completeness of the original source and the capabilities of automated information extractors, which are far from perfect \cite{mitchell2018, bordes2014b}. Additionally, KGs that are built manually, either by their creators or by a crowdsourced process, tend to be much smaller in size \cite{miller1995}. Due to these reasons, there exists a large number of incomplete KGs under active use today, and a need to complete them \cite{shrivastava2017, krishnan2018, pittman2017, noy2019, singhal2012}.

% In the literature, there are different proposals to address the problem of
% completing Knowledge Graphs, e.g., \cite{jiang2016, galarraga2015, wang2015, kuzelka2019, yang2017, sadeghian2019, minervini2018, wei2015, garcia-duran2015, yang2014, tay2017, trouillon2016, lin2015, do2018, gardner2015, nastase2019, gu2015, toutanova2016, jiang2017, lei2019, bansal2019a2n, wang2019}. Unfortunately, these proposals have a number of drawbacks that hinder their applicability in practice. Consequently, it is still necessary to research on the field of Knowledge Graph completion, which is our purpose in this dissertation.

\section{Problems}\label{sec:moti-problems}
\begin{itemize}
    \item They require embeddings to work.
    \item Only specific solutions exist for a particular approach (no general purpose)(no open source).
    \item little to no reason of included information (low explainability)
    \item something something reinforcement learning?
\end{itemize}
% Completing Knowledge Graphs is not a trivial task and, if not performed correctly, it may reduce the quality of the knowledge contained in them. In this section, we present the problems that must be addressed by proposals that perform this task in order to be useful in practice. These problems are as follows:

% \begin{description}
    % \item[(P1) To rely on embedded representations of entities and/or relations:] Many of the current proposals rely on generating, or being provided with, embedded representations of the entities and/or relations in a Knowledge Graph. These embedded representations are more compact versions of the elements they represent, usually as a vector that encodes the position of an entity or relation in an N-dimensional space. Embedded representations can be useful because they can capture meaningful semantic similarities between different elements, however, their use hinders the practical application of a proposal. Those proposals that require that embedded representations be provided to them are no longer stand-alone, they instead have a strong dependence with the tools that generates the embeddings and the quality of them. Even those that generate them themselves suffer from another issue: the need to completely remake them ---incurring in a very high computational cost--- whenever a new entity or relation is added to the KG, which is an event that tends to happen frequently \cite{mitchell2018,dong2014}. \\
    
    % \item[(P2) To depend on external sources of information:] An external source of information can be a structured or semi-structured repository of information, an information retrieval or query system, another Knowledge Graph or, in general any means for automatically accessing information besides the Knowledge Graph that is to be completed. Depending on these external sources introduces a single point of failure in the KG completion process that is not admissible in many practical and commercial applications. Furthermore, depending on the nature of the information source, its contents or means of access may change without any previous warning, rendering the KG completion proposal ineffective or, at least, requiring extensive maintenance.\\
    
    % \item[(P3) To need user-provided data or supervision:] Knowledge Graphs can store large amounts of information; some of the most well-known KGs used nowadays contain millions of triples, and some of them can reach even higher orders of magnitude \cite{singhal2012}. For this reason, it is not reasonable to rely on any sort of human-provided information in order to complete them. The required volume of human input for a standard Knowledge Graph would be unattainable for just a few human experts, and using a crowdsourcing process would greatly decrease the quality of the information that is fed to the KG completion process.\\
    
    % \item[(P4) To not have any means to automatically generate new knowledge:] Many of the existing KG completion proposals in the literature are able to determine whether a triple is correct or not, which is undoubtedly an important step, but they do not have any mechanisms to autonomously generate new knowledge. Instead, they rely on a set of possible facts being passed on to them for evaluation, but they do not specify how this set of facts should be created. Other proposals have a limited way to do this, by suggesting which entities are more likely to appear in a triple along with another given entity and relation. Although it is an improvement, it is still not practical to assess all possible pairs of entities and relations in a KG.\\
    
    % \newpage%\todo{quitar?}
    % \item[(P5) To not be applicable to large Knowledge Graphs:] As mentioned previously, most Knowledge Graphs contain large volumes of information. Even if a proposal relies entirely on information present in the graph, some of them use approaches that are known not to be scalable enough to be applied to some of the most popular Knowledge Graphs available nowadays.\\

% \end{description}

\section{Analysis of current solutions}\label{sec:moti-analysis}

Here add some tables similar to the ones already presented in ohter papers which evaluate multiple proposals in the topic, then compare why these approaches are not complete and why ours makes strides towards filling the gaps.

Most notably, terminal rewards and retropropagation, use of REINFORCE algorithm, little in the way of Actor-Critic approaches, precomputation and lack of usability.

An enumeration of references is also a possibility for this section, adding the benefits of each proposal in an incremental way and what they are missing.

% There already exist a number of proposals for completing Knowledge Graphs in the literature. In Table~\ref{table:proposals}, we summarize them and the problems they suffer from, and in the following, we discuss them in more detail:

% \begin{table}[!htp]
    \begin{center}
        \renewcommand{\arraystretch}{2.0}
    \begin{tabular}{>{\raggedright\arraybackslash} M{5cm} | M{1cm} | M{1cm} |  M{1cm} |  M{1cm} |  M{1cm}}
    
    \specialrule{1.2pt}{3pt}{3pt}
    \centering \textbf{Proposal} & \textbf{P1} & \textbf{P2} & \textbf{P3} & \textbf{P4} & \textbf{P5}  \\
    \specialrule{1.2pt}{3pt}{3pt}

    \citet{bordes2013} & \no & \yes & \yes & \no & \yes \\ \hline % TransE 
    \citet{galarraga2015} & \yes & \yes & \yes & \yes & \no \\ \hline % AMIE
    \citet{gardner2015} & \yes & \yes & \no & \no & \yes \\ \hline  % SFE
    \citet{guo2016} & \no & \yes & \no & \no & \yes \\ \hline  % KALE
    \citet{jiang2016}& \yes & \no & \yes & \yes & \no \\ \hline % ILP
    \citet{kazemi2018} & \no & \yes & \yes & \no & \yes \\ \hline % SimplE
    \citet{lao2011} & \yes & \yes & \yes & \no & \yes \\ \hline % PRA
    \citet{lin2015} & \no & \yes & \yes & \no & \yes \\ \hline % TransR
    
    \citet{nickel2011} & \no & \yes & \yes & \yes & \no \\ \hline % RESCAL
    \citet{shi18} & \no & \no & \yes & \yes & \yes \\ \hline % ConMask
    \citet{trouillon2016} & \no & \yes & \yes & \yes & \no \\ \hline % ComplEx
    
    \citet{wang2014} & \no & \yes & \yes & \no & \yes \\% TransH
    \specialrule{1.2pt}{3pt}{3pt}
    
    
    
    
    
    
    



    \end{tabular}

    \vspace{.5cm}
    {
        \flushleft
        P1 = To rely on embedded representations of entities and/or relations; P2 = To depend on external sources of information; P3 = To need user-provided data or supervision; P4 = To not have any means to automatically generate new knowledge; P5 = To not be applicable to large Knowledge Graphs.\\
        
        \flushleft
        A \yes{} means that the proposal is free from a problem, while \no{} means that it is present.
    }

    \caption{Comparison of current proposals for KG completion}
    \label{table:proposals}
    \end{center}
\end{table}

% \citet{bordes2013} devised a proposal that learns embedded representations of the entities in a Knowledge Graph, in order to place them in an N-dimensional space while transforming semantical similarities into physical closeness in said space. In this proposal, the correctness of a triple can be checked by evaluating the relative positions of its two entities in the embedded space.

% \citet{galarraga2015} proposed a rule extractor that is able to capture common patterns in a Knowledge Graph using Inductive Logic Programming (ILP), and express them using first-order rules. Once these rules have been mined, they can be applied to materialize new knowledge in the KG.

% \citet{gardner2015} introduced a technique that defines a series of features to characterize the path between two entities, and then analyzes a large number of said paths to learn to identify a possible direct connection between the entities. However, it requires the manual introduction of an ``Alias'' relation, which indicates that two entities in a KG refer to the same concept in the real world.

% \citet{guo2016} presented a proposal that combines the use of entity embeddings and logical rules. It provides a shared framework in which rules and embeddings can directly interact with each other. This is done by representing the triples in a KG mathematically, and defining a series of operators. The semantic information present in the entity embeddings helps expand the predictive capabilities of the rules that this proposal produces.

% \citet{jiang2016} proposed another approach that uses ILP to find and exploit rules in a Knowledge Graph, more particularly, by analyzing the intervals of validity of the facts contained within it and reasoning when other related facts will start or stop being valid. This requires the introduction of temporal annotations in the KG which, generally, must be manually provided.

% \citet{kazemi2018} also leverage entity and relation embeddings, and propose adding an extra inverse relation for every one that is already present in a KG. This allows their proposal to reach a higher degree of expressivity, while its simple embedding model allows it to be applied to large KGs.

% \citet{lao2011} presented a technique that uses random walks to traverse the space in the KG between the two entities of a triple. By analyzing examples of correct and incorrect triples, their proposal is able to learn whether two entities should be connected according to the possible paths that can be traced between them.

% \citet{lin2015} proposed a different use of embedding spaces, by defining two groups of them, one exclusively for entities and one for every different relation present in a KG. Their model is able to express triples as transformations between relation spaces through the use of projection matrices. The increased number of embedding spaces makes it more computationally complex, but also more effective.

% \citet{nickel2011} suggested using tensors to represent a Knowledge Graph, and then factorizing those tensors to obtain more compact representations of the knowledge in them. Their approach works by using a third-order binary tensor that holds information about which entities are connected in the graph and through which relation, and then performing a series of mathematical operations that result in another tensor, with confidence levels for any possible fact that could be introduced in the KG, given its current entities and relations. However, the size of the tensor scales quadratically with respect to the number of entities, making it a poor choice for large KGs.

% \citet{shi18} introduced a proposal that uses not only the embeddings of an entity, but also of its textual description, to look for additional levels of semantic similarity. Additionally, their proposal does not rely on generating negative examples, like most other existing state-of-the-art proposals do. However, it relies on external sources of information to retrieve the descriptions of the entities.

% \citet{trouillon2016} proposed a technique that factorizes the tensor representation of a KG, but instead of binary values, it uses complex numbers. This can be seen as generating two separate tensors: one which contains the real parts of the values, and another one that contains the imaginary parts. The usage of complex numbers allows it to simplify some of its internal calculations, making it slightly more efficient.

% \citet{wang2014} proposed using translation on hyperplanes to change the embedded representation of an entity, depending on which relation is being used to link it with another entity. However, these translations, just like the overall embedding space, must be re-learned whenever a change in the KG occurs.

\section{Discussion}\label{sec:moti-discussion}

This section is not on both Agu's and Inma's thesis(es?), a mixed approach is best I believe, present a table that evaluate each of the proposals shortcomings and a brief explanation for each of them is also good.

In summary remove this section and provide all information on the previous one.
% The previous proposals have problems that hinder their applicability in practice. Regarding the use of embeddings (P1), most of the existing KG completion techniques in the literature rely on them to some extent \cite{bordes2013,guo2016,kazemi2018,lin2015,nickel2011,shi18,trouillon2016,wang2014}. This is problematic because, as discussed, most KGs are in constant expansion, and adding any entity or relation to the KG must trigger a re-computation of said embeddings, which is not feasible in practice.

% \citet{jiang2016} and \citet{shi18} propose techniques that depend on outside information (P2), making them vulnerable to changes in the sources of external information. Furthermore, the proposals made by \citet{gardner2015} and \citet{guo2016} are not fully automated, and require manual intervention by the users (P3).

% Regarding the automated generation of new knowledge (P4), there are few techniques that are able to do it independently. The approaches proposed by \citet{galarraga2015} and \citet{jiang2016} provide a straight-forward way to achieve this, since they generate first-order logical rules that can be used to materialize new knowledge in the KG. The tensor factorization approaches introduced by \citet{nickel2011} and \citet{trouillon2016} also enable this by traversing the resulting tensor, which contains confidence scores for all possible facts, and adding those whose score exceeds a certain threshold (although this may be time-consuming given the size of the tensors). Finally, \citet{shi18} use a simplified way of generating candidate triples, by generating and examining those in which the left-hand entity and relation already appear together somewhere else in the KG.

% Most existing proposals can be applied to large Knowledge Graphs (P5), although there are some exceptions. The proposals made by \citet{galarraga2015} and \citet{jiang2016} rely upon rule mining through ILP, which is known to scale poorly to large collections of facts \cite{shen2022overview}. Additionally, tensor factorization techniques, such as those proposed by \citet{nickel2011} and \citet{trouillon2016}, need to generate tensors that contain $R \cdot E^2$ elements, where $R$ is the number of distinct relations in a KG and $E$ is the amount of different entities. This size can quickly become unmanageable when the KG contains more than a few thousand entities.

\section{Our proposal}\label{sec:moti-proposal}
In this dissertation, we present a proposal for Knowledge Graph reasoning SpaceRL which...


% To determine the correctness of a triple, we introduce CAFE, our triple classification technique. It relies solely on path and entity neighborhood information present in the graph, which, contrary to embeddings, does not require any pre-computation or significant re-generation when the graph changes (P1). It is also able to operate taking only a Knowledge Graph as input, without any additional dependencies on external sources of information (P2). CAFE works by transforming the triples in a KG into a set of labeled feature vectors, using a novel set of context-aware features, and then learning to differentiate between correct and incorrect triples in a completely automated manner, without any user intervention (P3).

% We also present CHAI, our proposal for automatically generating candidate triples. CHAI is able to materialize sets of possible facts of a reasonable size, that include most of the information that is missing in a KG. These facts can be passed on to a triple classification technique such as CAFE, thus solving the problem of generating new knowledge (P4).

% Finally, to prove the applicability of our proposal, we introduce SciCheck, an extension of CAFE specifically tailored to scientific Knowledge Graphs. We show that SciCheck can be applied to AI-KG, a large scientific KG with over 14 million triples, yielding very satisfactory results and taking considerably less time than other existing approaches in the literature (P5).

\section{Summary}\label{sec:moti-summary}
Summary of section.
% In this chapter, we have motivated the reason for this dissertation. We have analyzed the problems of completing Knowledge Graphs and the current proposals in the literature to carry out this task, and we have concluded that none of these proposals solves all of the presented problems at a time.