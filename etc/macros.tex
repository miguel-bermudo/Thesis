% Citas y abstracts de los capítulos
\newcommand{\chapterQuote}[2]{
    \vspace{-1.5cm}
    \epigraph{#1}{#2}
}

\newcommand{\chapterAbstract}[2]{
    % Seguramente haya alguna forma de pasar sólo 1 argumento
    % y extraer automáticamente la primera letra para pasársela
    % al lettrine. Los ejemplos de cosas parecidas que he
    % encontrado son todos feísimos y usan latex oscuro, así
    % que paso, no me merece la pena el esfuerzo. Si tú, usuario
    % futuro de esta plantilla, consigues hacerlo así, cambia
    % este comando y siéntete realizado con tu vida.
    \vspace{2cm}
    \lettrine{#1}{ }#2
    \newpage
}

\makeatletter
\newcommand{\monthyeardate}{%
  \DTMenglishmonthname{\@dtm@month} \@dtm@year
}

\newcommand{\monthyearcover}{%
  \DTMenglishmonthname{\@dtm@month}, \@dtm@year
}

\newcommand*{\rom}[1]{\expandafter\@slowromancap\romannumeral #1@}

\newcommand{\romanyear}{%
    \rom{\@dtm@year}
}
\makeatother

\newcommand{\hypthesis}[1]
{
    \begin{quote}
        \textit{#1}
    \end{quote}
}

% avoid having to prepend ~ to \cite and \ref all the time
% to prevent the reference from being placed in the next line
\let\oldcite\cite
\renewcommand\cite{\nolinebreak\oldcite}

\newcommand{\tripleSty}[1]{\textit{#1}}

%formatting for code excerpts
\newcommand{\code}[1]{\texttt{#1}}

% Para que figuras grandes no se coman una pagina entera
% https://tex.stackexchange.com/questions/68516/avoid-that-figure-gets-its-own-page
\renewcommand{\floatpagefraction}{.9}

%%%%%%%%%%%%%%%%%%%%%%%%%%%%%%%%%%%%%%%%%%%%%%%%%%%%%%%
%%% A partir de aquí hacia abajo son cosas que he definido yo, se pueden borrar

% this is so the text f's in math mode don't have spacing around them
% by using {\ff}
\newcommand{\ff}{\mkern-2mu f\mkern-3mu}

%tipo de columna para las tablas
\newcolumntype{M}[1]{>{\centering\arraybackslash}m{#1}}

% shortcuts para formulas
\newcommand{\fancy}[1]{\ensuremath{\mathcal{#1}}}
\newcommand{\Eset}{\fancy{E}}
\newcommand{\Rset}{\fancy{R}}
\newcommand{\Tset}{\fancy{T}}
\newcommand{\Sset}{\fancy{C}}
\newcommand{\triple}{\ensuremath{(s, r, t)}}
\newcommand{\negtriple}{\ensuremath{(s, r, t')}}
\newcommand{\KG}{\ensuremath{\textmd{\textup{\fancy{K}\fancy{G}}}}}
\newcommand{\KGlong}{\ensuremath{(\Eset{}, \Rset{}, \Tset{})}}
\newcommand{\subKG}[2]{\ensuremath{\KG_{#1}^{#2}}}
\newcommand{\subKGdef}{\subKG{n}{n}}
\newcommand{\nearby}[2]{\Eset{}_{#1}^{#2}}
\newcommand{\kgpath}[3]{\ensuremath{path(#1, #2, #3)}}
\newcommand{\kgpathl}[3]{\ensuremath{path_{#3}(#1, #2)}}
\newcommand{\kgpathP}[3]{\ensuremath{\fancy{P}(#1, #2, #3)}}


% entornos y comandos para definiciones y cosas
\newcommand{\defin}[2]{
\begin{definition}\label{#1}{\textbf{#1:}}
#2
\end{definition}
}

\newcommand{\candfunc}[2]{
    \textbf{#1}: #2
}

\newcounter{featgroup}
\newcommand{\featgroup}[6]{
    \stepcounter{featgroup}
    \textbf{\texorpdfstring{Feature group $f_{\arabic{featgroup}}$: \textit{#1}}{Feature group f\arabic{featgroup}: #1}} Features in this group are computed as: 
    \begin{equation*}
    \begin{gathered}
        f_{\arabic{featgroup}}(#2) : \triple \mapsto #3
    \end{gathered}
    \end{equation*}
    In the example shown in Figure \ref{fig:kg-potter}, $f_{\arabic{featgroup}}($\textit{#5}$)$ applied to the $\textit{example}$ triple is #6.
}

\newcounter{scifeatcount}
\newcommand{\scifeature}[2]{
    \stepcounter{scifeatcount}
    \textbf{Feature $f_{\arabic{scifeatcount}}$: #1}. #2
}

\newcommand{\roberta}{RoBERTa}


\definecolor{tickgreen}{HTML}{009B55} 
\definecolor{crossred}{HTML}{850000} 
\newcommand{\yes}{\textcolor{tickgreen}{\textbf{\checkmark}}}
\newcommand{\no}{\textcolor{crossred}{\text{\sffamily X}}}