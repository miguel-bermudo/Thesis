% Fuente
\usepackage[sc]{mathpazo}

\linespread{1.2} % Espacio entre líneas
\setlength{\parindent}{20pt} % Indentación de primera línea en párrafos

% Configurar la fuente para la primera letra de los capítulos
\renewcommand{\LettrineFontHook}{\Zallmanfamily}

% Estilo para las citas al principio de los capítulos
\epigraphsize{\small\itshape}
\setlength{\epigraphwidth}{.6\textwidth}
\renewcommand{\textflush}{flushepinormal}
\setlength\epigraphrule{0pt}

% Tamaños y estilos de elementos en la TOC
\DeclareTOCStyleEntry[
    linefill=\bfseries\TOCLineLeaderFill,
    beforeskip=12pt,
    entrynumberformat=\chapterprefixintoc,
    entryformat=\chaptertocformat,
    pagenumberformat=\chaptertocformat,
    dynnumwidth
]{tocline}{chapter}

\DeclareTOCStyleEntry[
    % linefill=\bfseries\TOCLineLeaderFill,
    beforeskip=30pt,
    entrynumberformat=\chapterprefixintoc,
    entryformat=\parttocformat,
    pagenumberformat=\partpagetocformat,
    numwidth=0pt
]{tocline}{part}

\newcommand\chapterprefixintoc[1]{#1}%
\newcommand\chaptertocformat[1]{\larger[1]{\textbf{#1}}}%
\newcommand\parttocformat[1]{\larger[2]{\textbf{#1}}}%
\newcommand\partpagetocformat[1]{} % Don't print the page number for parts


\definecolor{ColorRedUS}{cmyk}{0,1,0.65,0.34}
\definecolor{ColorYellowUS}{cmyk}{0,0.08,0.27,0}
\definecolor{ColorFontETSII}{cmyk}{0.51,0.39,0,0.68}

\setcounter{DefaultLines}{3}

\newcommand*{\ecap}[1]{\begingroup\obeyspaces\etospace{#1}{}}
\newcommand*{\etospace}[3]{\ifthenelse{\equal{#3}{\space}}%
{\endgroup\lettrine[findent=10pt,nindent=0pt]{#1}{}#2#3}{\etospace{#1}{#2#3}}}

%%% Comandos para cajas
%Width, backcolor, linecolor, textcolor, text
\newcommand{\drCoverBox}[5]{
    \begin{tcolorbox}[
        enhanced,
        width=#1,
        arc=0mm,
        colframe=#2,
        colback=#2,
        borderline horizontal={4pt}{0pt}{#3},
        halign=center,
        code={\doublespacing}
        ]
        {\color{#4}\setstretch{2.2}#5}
        \end{tcolorbox}
}


%Width, backcolor, linecolor, textcolor, title, text
\newcommand{\drPartBox}[6]{
    \begin{tcolorbox}[
        enhanced,
        width=#1,
        arc=0mm,
        colframe=#2,
        colback=#2,
        borderline horizontal={4pt}{0pt}{#3},
        title=#5,
        arc=0mm,
        coltitle=#3,
        boxed title style={size=small,colback=#2,colframe=#3},
        attach boxed title to top right={yshift=-\tcboxedtitleheight/2, xshift=-12pt},
        halign=center]
        {\color{#4}#6}
    \end{tcolorbox}
    }

%Width, backcolor, linecolor, textcolor, text
\newcommand{\drHeadBox}[7]{
    \begin{tcolorbox}[
        enhanced,
        width=#1,
        title=#5,
        borderline north={4pt}{0pt}{#2},
        arc=0mm,
        colframe=white,
        colback=white,
        coltitle=#3,
        boxed title style={colback=#2,colframe=#2,arc=6pt},
        attach boxed title to top #6={#7},
        halign=center]
        %{\color{#3}}
    \end{tcolorbox}
}

 \newcommand{\drRightBox}[5]{
     \drHeadBox{#1}{#2}{#3}{#4}{#5}{right}{yshift=-11pt, xshift=-25mm}
 }

 \newcommand{\drLeftBox}[5]{
     \drHeadBox{#1}{#2}{#3}{#4}{#5}{left}{yshift=-11pt, xshift=25mm}
 }
 
%Width, backcolor, linecolor, textcolor, text
\newcommand{\drHeadBall}[7]{
\begin{tcolorbox}[
    enhanced,
    width=#1,
    title=#5,
    borderline north={4pt}{0pt}{#2},
    colframe=white,
    colback=white,
    coltitle=#3,
    boxed title style={colback=#2,colframe=#2,circular arc},
    attach boxed title to top #6={#7}, %right={yshift=-11pt, xshift=-12pt},
    halign=center,
    valign=center]
    {\color{#3}}
\end{tcolorbox}
}

\newcommand{\drRightBall}[5]{
    \drHeadBall{#1}{#2}{#3}{#4}{#5}{right}{yshift=-11pt, xshift=-12pt}
}

\newcommand{\drLeftBall}[5]{
    \drHeadBall{#1}{#2}{#3}{#4}{#5}{left}{yshift=-11pt, xshift=12pt}
}


%%
%% Page head (even and odd)
%%
\newcommand*\drHeadEven{%
    \begin{adjustwidth}{-25mm}{25mm}
        \drLeftBox{\paperwidth}{ColorYellowUS}{ColorRedUS}{ColorRedUS}{\itshape\nouppercase{\leftmark}}
    \end{adjustwidth}
}

\newcommand*\drHeadOdd{%
    \begin{adjustwidth}{-25mm}{25mm}
        \drRightBox{\paperwidth}{ColorYellowUS}{ColorRedUS}{ColorRedUS}{\itshape\nouppercase{\rightmark}}
    \end{adjustwidth}
}

%% Page feet (even and odd)
\newcommand*\drFootEven{%
    \begin{adjustwidth}{-25mm}{25mm}
        \drLeftBox{\paperwidth}{ColorYellowUS}{ColorRedUS}{ColorRedUS}{\thepage}
    \end{adjustwidth}
}

\newcommand*\drFootOdd{%
    \begin{adjustwidth}{-25mm}{25mm}
        \drRightBox{\paperwidth}{ColorYellowUS}{ColorRedUS}{ColorRedUS}{\thepage}
    \end{adjustwidth}
}

\newcommand*\drFootOddTitle{%
    \begin{adjustwidth}{-25mm}{25mm}
        \drRightBox{\paperwidth}{ColorRedUS}{ColorYellowUS}{ColorRedUS}{\thepage}
    \end{adjustwidth}
}

%% Page style for general pages 
\fancypagestyle{drStyle}{%
    \fancyhf{}
    \fancyhead[CE]{\drHeadEven}
    \fancyhead[CO]{\drHeadOdd}
    \fancyfoot[CO]{\drFootOdd}
    \fancyfoot[CE]{\drFootEven}
    \renewcommand{\headrulewidth}{0pt}
    \renewcommand{\footrulewidth}{0pt}
}

%% Page style for plain pages (chapters and parts)
\fancypagestyle{plain}{%
    \fancyhf{}
    \fancyfoot[CO]{\drFootOddTitle}
    \fancyfoot[CE]{\drFootEven}
    %\renewcommand{\headrulewidth}{0pt}
    %\renewcommand{\footrulewidth}{0pt}
}

%% Page style for title pages
\fancypagestyle{title}{%
 \fancyhf{}
 \fancyhead[CE]{}
 \fancyhead[CO]{}
}

%% Page headings for empty pages
\fancypagestyle{empty}{%
    \fancyhf{}
    \renewcommand{\headrulewidth}{0pt}
    \renewcommand{\footrulewidth}{0pt}
}


%% Defining Chapter Heading Style
%\assignpagestyle{\chapter}{title}
%\titleformat{command}[shape]{format}{label}{sep}{before-code}[after-code]
\newcommand*\chapterlabel{}
\titleformat{\chapter}
  {\gdef\chapterlabel{}}
  {\gdef\chapterlabel{\thechapter\ }}
  {0pt}
  { 
    \ifx \chapterlabel \empty %Indices, preámbulo
        % \hspace*{\fill}
        {\normalfont\huge\bfseries\scshape\color{ColorRedUS}#1}\\
        {\color{ColorRedUS}\rule{\textwidth}{2pt}}
    \else %Resto de capítulos
        \begin{center}
            \drPartBox{0.9\linewidth}{white}{ColorRedUS}{ColorRedUS}{
                {\normalfont\huge\bfseries\scshape\color{ColorRedUS}Chapter \chapterlabel}
                }{
                    \vspace{24pt}
                    {\normalfont\huge\bfseries\scshape\color{ColorRedUS}#1}
                    \vspace{24pt}
                    }        
        \end{center}
    \fi
  }

%% Defining Part Heading Style
%\newcommand*\partlabel{\thispagestyle{plain}}
\let\Oldpart\part
\newcommand{\parttitle}{}
\renewcommand{\part}[1]{\Oldpart{#1}\renewcommand{\parttitle}{#1}}
 \titleformat{\part}
   {\gdef\partlabel{}
    \normalfont\Huge\bfseries\scshape} %Format
   {\gdef\partlabel{\thepart\ }} % Label
   {0pt} %Sep
   {
        \thispagestyle{empty}
       \begin{center}
            \drPartBox{0.7\linewidth}{white}{ColorRedUS}{ColorRedUS}{Part \partlabel}{
                \vspace{24pt}
                \parttitle{#1}
                \vspace{24pt}
            }
        \end{center}
   }

\setitemize{noitemsep,topsep=0pt,parsep=0pt,partopsep=0pt}

\newtheoremstyle{definition}% % see p. 10 of user manual of amsthm package
{\topsep}%  Space above
{\topsep}%  Space below
{}%     Body font
{}%     Indent amount
{\bfseries}% Theorem head font
{}%    Punctuation after theorem head -- none
{.5em}% Space after theorem head
{\thmname{#1}\thmnumber{ #2}\thmnote{ (#3)}}% Theorem head spec 

\usepackage[
    %showframe,
    includeheadfoot,
    headheight = 25mm,
    headsep = 0mm,
    footskip = 5mm,%\dimexpr\headsep+\ht\strutbox\relax,
    tmargin = 0mm,
    bmargin = 15mm,%\dimexpr2mm+2\ht\strutbox\relax,
    lmargin = 25mm,
    rmargin = 25mm
]{geometry}

%listings

\definecolor{codegreen}{rgb}{0,0.6,0}
\definecolor{codegray}{rgb}{0.5,0.5,0.5}
\definecolor{codepurple}{rgb}{0.58,0,0.82}
\definecolor{editorOcher}{rgb}{0.8, 0.3, 0} % #FF7F00 -> rgb(239, 169, 0)
\definecolor{editorGreen}{rgb}{0, 0.5, 0} % #007C00 -> rgb(0, 124, 0)

%JSON
\newcommand\jsonkey{\color{editorOcher}}
\newcommand\jsonvalue{\color{editorGreen}}
\newcommand\jsonnumber{\color{editorOcher}}

\lstdefinestyle{listingstyle}{
    backgroundcolor=\color{white},  
    keywordstyle=\bfseries\color{blue},
    numberstyle=\tiny\color{codegray},
    stringstyle=\color{editorGreen},
    commentstyle=\color{codegray},
    basicstyle=\ttfamily\footnotesize\color{black},
    breakatwhitespace=false,         
    breaklines=true,                 
    captionpos=b,                    
    keepspaces=true,                 
    numbers=left,                    
    numbersep=5pt,                  
    showspaces=false,                
    showstringspaces=false,
    showtabs=false,                  
    tabsize=2,
    frame=tb,
    keywords=[2]{True,False},
    literate=%
*{0}{{{\color{editorOcher}0}}}1
{1}{{{\color{editorOcher}1}}}1
{2}{{{\color{editorOcher}2}}}1
{3}{{{\color{editorOcher}3}}}1
{4}{{{\color{editorOcher}4}}}1
{5}{{{\color{editorOcher}5}}}1
{6}{{{\color{editorOcher}6}}}1
{7}{{{\color{editorOcher}7}}}1
{8}{{{\color{editorOcher}8}}}1
{9}{{{\color{editorOcher}9}}}1,
}


\lstdefinelanguage{python}{
  morekeywords={True, False},
  morecomment=[l]\#,
  morestring=[b]",
  morestring=[b]'
}

% latex black magic to make json color highlighting work
% i wont even pretend i have the slightest idea about how this works
% kudos to https://tex.stackexchange.com/a/195540
% flip the switch if a colon is found in Pmode
\makeatletter
\newif\ifisvalue@json

\lstdefinelanguage{json}{
    tabsize             = 4,
    showstringspaces    = false,
    keywords            = {false,true},
    alsoletter          = 0123456789.,
    morestring          = [s]{"}{"},
    stringstyle         = \jsonkey\ifisvalue@json\jsonvalue\fi,
    MoreSelectCharTable = \lst@DefSaveDef{`:}\colon@json{\enterMode@json},
    MoreSelectCharTable = \lst@DefSaveDef{`,}\comma@json{\exitMode@json{\comma@json}},
    MoreSelectCharTable = \lst@DefSaveDef{`\{}\bracket@json{\exitMode@json{\bracket@json}},
    basicstyle          = \ttfamily
}

% enter "value" mode after encountering a colon
\newcommand\enterMode@json{%
    \colon@json%
    \ifnum\lst@mode=\lst@Pmode%
        \global\isvalue@jsontrue%
    \fi
}

% leave "value" mode: either we hit a comma, or the value is a nested object
\newcommand\exitMode@json[1]{#1\global\isvalue@jsonfalse}

\lst@AddToHook{Output}{%
    \ifisvalue@json%
        \ifnum\lst@mode=\lst@Pmode%
            \def\lst@thestyle{\jsonnumber}%
        \fi
    \fi
    %override by keyword style if a keyword is detected!
    \lsthk@DetectKeywords% 
}

\makeatother

\lstset{style=listingstyle}
\lstset{columns=fullflexible}